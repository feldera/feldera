\section{Recursive queries}\label{sec:recursion}

Recursive queries are very useful in many applications.  For
example, graph algorithms such as graph reachability or transitive
closure are naturally expressed using recursive queries.  We introduce
two simple \dbsp stream operators that are used for expressing
recursive query evaluation.  These operators allow us to build
circuits implementing looping constructs, which are used to iterate
computations until a fixed-point is reached (i.e., the output of some
operator does not change anymore).

\subsection{Creating and destroying streams}

%\begin{definition}\label{def:zae}
%We say that a stream $s \in \stream{A}$ is \defined{zero almost-everywhere} if it has a finite
%number of non-zero values, i.e., there exists a time $t_0 \in \N$
%s.t. $\forall t \geq t_0 . s[t] = 0$.
%\noindent Denote the set of streams that are zero almost everywhere
%by $\streamf{A} \subset \stream{A}$.
%\end{definition}

% \paragraph{Stream introduction}

The \defined{delta function} $\delta : A \rightarrow \stream{A}$
produces a stream from a scalar value.  Given an input value $x$, the
output stream is $x$ followed by an infinite number of zeros.  The
input of $\delta$ has a single arrow, while the output has a double
arrow.
%
%$$\delta(v)[t] \defn \left\{
%\begin{array}{ll}
%  v & \mbox{if } t = 0 \\
%  0_A & \mbox{ otherwise}
%\end{array}
%\right.
%$$
%\ifstreamexamples
%For example, $\delta(5)$ is the stream $\sv{5 0 0 0 0}$.
%\fi
\begin{center}
\begin{tikzpicture}[auto,node distance=1cm,>=latex]
    \node[] (input) {\fbox{x}};
    \node[block, right of=input] (delta) {$\delta$};
    \node[right of=delta, node distance=2.2cm] (output) {\sv{x 0 0 0 0}};
    \draw[->] (input) -- (delta);
    \draw[->>] (delta) -- (output);
\end{tikzpicture}
\end{center}

% \paragraph{Stream elimination}

We define the function $\int : \stream{A} \rightarrow A$.  Its input
stream is required to eventually reach the value 0 and never change
afterwards.  This function just sums up all the values in the input
stream and returns a single result when it encounters the first 0 in
the input stream.  Notice that the input is a double arrow, while the
output is a single arrow.  E.g.,:

\begin{center}
\begin{tikzpicture}[auto,node distance=1cm,>=latex]
    \node[] (input) {$\sv{1 2 3 0 0}$};
    \node[block, right of=input, node distance=2.2cm] (S) {$\int$};
    \node[right of=S] (output) {$\fbox{6}$};
    \draw[->>] (input) -- (S);
    \draw[->] (S) -- (output);
\end{tikzpicture}
\end{center}
%
\noindent (This function is also an integrator; its relationship to
the $\I$ operator is the same one as the relationship of the definite
integral~\cite{integral} to the indefinite
integral~\cite{antiderivative} in mathematics.)

$\delta$ and $\int$ are both linear.

%$\delta$ is the left inverse of $\int$, i.e.: $\int \circ \; \delta = \id_A$.
%\begin{proposition}
%$\delta$ and $\int$ are LTI.
%\end{proposition}

% \paragraph{Nested time domains}

So far we have used a tacit assumption that ``time'' is common for all
streams in a program.  For example, when we add two streams, we assume
that they use the same ``clock''.  However, the $\delta$ operator
creates a stream with a ``new'', independent time dimension.  We
require \emph{well-formed circuits} to ``insulate'' nested time
domains by ``bracketing'' them between a $\delta$ and an $\int$
operator:
%
\begin{center}
\begin{tikzpicture}[auto,node distance=1cm,>=latex]
    \node[] (input) {$i$};
    \node[block, right of=input] (delta) {$\delta$};
    \node[block, right of=delta] (f) {$S$};
    \draw[->] (input) -- (delta);
    \draw[->>] (delta) -- (f);

    \node[block, right of=f] (S) {$\int$};
    \node[right of=S] (output) {$o$};
    \draw[->>] (f) -- (S);
    \draw[->] (S) -- (output);
\end{tikzpicture}
\end{center}
\vspace{-1ex}
$S$ is a streaming operator, but the entire circuit implements a
scalar function, as shown by the single arrowheads.

%\begin{proposition}
%If $Q$ is time-invariant, the circuit above has the zero-preservation
%property: $\zpp{\int \circ\; Q \circ \delta}$.
%\end{proposition}

\subsection{Implementing recursive queries}\label{sec:datalog}

We describe the implementation of recursive queries in \dbsp.  SQL can
only express very limited recursive queries, so here we model Datalog
queries.  In general, a Datalog program defines a set of mutually
recursive relations.

We describe the algorithm to build DBSP circuits for the simple case
of a single-input, single-output recursive query.  The input of our
algorithm is a Datalog query of the form $O = R(x, O)$, where $R$ is a
\textbf{relational, non-recursive} query, producing a set as a result,
but whose output $O$ is also an input.  The output of the algorithm is
a \dbsp circuit which evaluates this recursive query producing output
$O$ when given the input $x$.  In this section we build a
non-incremental circuit, which evaluates the Datalog query; in
\refsec{sec:nested} we derive the incremental version of this circuit.

\begin{algorithm}[recursive queries]\label{algorithm-rec}
\noindent
\begin{enumerate}[nosep, leftmargin=0pt, itemindent=0.5cm, label=\textbf{(\arabic{*})}]
\item Implement the non-recursive relational query $R$ as described in
    \secref{sec:relational} and Table~\ref{tab:relational}; this produces
    an acyclic circuit whose inputs and outputs are \zrs:
    \begin{center}
    \begin{tikzpicture}[auto,>=latex]
      \node[] (I) {$x$};
      \node[below of=I, node distance=.5cm] (O) {$O$};
      \node[block, right of=I] (R) {$R$};
      \node[right of=R] (o) {$O$};
      \draw[->] (I) -- (R);
      \draw[->] (O) -| (R);
      \draw[->] (R) -- (o);
    \end{tikzpicture}
    \end{center}
    \vspace{-2ex}
%
In all these diagrams we show input 0 of operator $R$ on the left, and
input 1 on the bottom.

\item Lift this circuit to operate on streams:
    \begin{center}
    \begin{tikzpicture}[auto,>=latex]
      \node[] (I) {$x$};
      \node[below of=I, node distance=.7cm] (O) {$O$};
      \node[block, right of=I] (R) {$\lift R$};
      \node[right of=R] (o) {$O$};
      \draw[->>] (I) -- (R);
      \draw[->>] (O) -| (R);
      \draw[->>] (R) -- (o);
    \end{tikzpicture}
    \end{center}
    \vspace{-2ex}

  Construct $\lift{R}$ by lifting each operator individually, using
  equation (**) in \refsec{sec:streams}.

\item Build a cycle, connecting the output to the corresponding
recursive input via a delay:

 \begin{center}
\begin{tikzpicture}[auto,>=latex, node distance=.8cm]
  \node[] (I) {$x$};
  \node[block, right of=I, node distance=1cm] (R) {$\lift R$};
  \node[right of=R, node distance=1.5cm] (O) {$O$};
  \node[block, below of=R, node distance=.8cm] (z) {$\zm$};
  \draw[->>] (I) -- (R);
  \draw[->>] (R) -- node(o) {} (O);
  \draw[->>] (o) |- (z);
  \draw[->>] (z) -- (R);
 \end{tikzpicture}
 \end{center}
 \vspace{-2ex}
\item ``Bracket'' the circuit as follows:
%
\begin{center}
\begin{tikzpicture}[auto,>=latex, node distance=.8cm]
  \node[] (Iinput) {$x$};
  \node[block, right of=Iinput] (ID) {$\delta$};
  \node[block, right of=ID] (II) {$\I$};
  \node[block, right of=II, node distance=1cm] (f) {$\lift{R}$};
  \node[block, right of=f, node distance=1.5cm] (D) {$\D$};
  \node[block, right of=D] (S) {$\int$};
  \node[right of=S] (output)  {$O$};
  \draw[->] (Iinput) -- (ID);
  \draw[->>] (ID) -- (II);
  \draw[->>] (II) -- (f);
  \draw[->>] (f) -- node (o) {} (D);
  \draw[->>] (D) -- (S);
  \draw[->] (S) -- (output);
  \node[block, below of=f, node distance=.8cm] (z) {$\zm$};
  \draw[->>] (o) |- (z);
  \draw[->>] (z) -- (f);
\end{tikzpicture}
    \vspace{-2ex}
\end{center}
\end{enumerate}
\end{algorithm}

The left input of $\lift{R}$ is an infinite stream of identical values
\sv{$x$ $x$ $x$ $x$ $x$}.  The feedback cycle in this circuit is a
\texttt{while} loop that iterates until no changes are observed (i.e.,
a fixed-point of $R$ is reached); the outputs produced by $\lift{R}$
will be: in sequence $R(x,0)$, $R(x,R(x,0))$, $R(x,R(x,R(x,0)))$,
etc..  The $\D$ operator yields the set of \emph{new changes} computed
by each iteration of the loop.  When the set of new changes becomes
zero, the fixed point has been reached.

Please note that this is \textbf{not} a streaming circuit: the input
and output arrows are both simple.  This is a circuit which receives a
single input value and produces a single corresponding output.  The
circuit uses streams internally to implement the fixed point
iteration.

A concrete example for a transitive closure query is Section~8.2 of
our technical report~\cite{tr}.

%\begin{theorem}[Recursion correctness]\label{theorem:recursion}
%If $\isset(\code{I})$, the output of the circuit above is
%the relation $\code{O}$ as defined by the Datalog semantics of given program
%$R$ as a function of the input relation \code{I}.
%\end{theorem}
%\label{proof-recursion}
%\begin{proof}
%Let us compute the contents of the $o$ stream, produced at the output
%of $R$.  We will show that this stream is composed
%of increasing approximations of the value of \code{O}.
%
%Define the following one-argument function: $S(x) = \lambda x . R(\code{I}, x)$.
%Notice that the left input of the $\lift{R}$ block is a constant stream
%with the value \code{I}.  Due to the stratified nature of the language,
%we must have $\ispositive(S)$, so $\forall x . S(x) \geq x$.
%We get the following system of equations:
%$$
%\begin{aligned}
%o[0] =& S(0) \\
%o[t] =& S(o[t-1]) \\
%\end{aligned}
%$$
%So, by induction on $t$ we have $o[t] = S^t(0)$, where by
%$S^t$ we mean $\underbrace{S \circ S \circ \ldots \circ S}_{t}$.
%$S$ is monotone; thus, if there is a time $k$ such that $S^k(0) = S^{k+1}(0)$, we have
%$\forall j \in \N . S^{k+j}(0) = S^k(0)$.  Applying a $\D$ to this stream
%will then produce a stream that is zero almost everywhere, and integrating
%this result will return the last distinct value in the stream $o$.
%
%This is essentially the definition of the semantics of a recursive Datalog relation:
%$\code{O} = \fix{x}{R(\code{I}, x)}$.
%\end{proof}

When $R$, the body of the loop, implements a Datalog programs
computing on a finite data domain, this program can be proven to
always terminate and compute the least fixed point that contains $x$.
For an arbirary function $R$, the resulting circuit may loop forever
for some inputs.

In fact, this circuit implements the standard Datalog
\defined{na\"{\i}ve evaluation} algorithm (e.g., see Algorithm~1 in
\cite{greco-sldm15}).  Notice that the inner part of the circuit is
the incremental form of another circuit, since it is sandwiched
between $\I$ and $\D$ operators.  Using the cycle rule we can
rewrite this circuit:
%
\begin{center}
\begin{tikzpicture}[auto,>=latex]
  \node[] (Iinput) {$x$};
  \node[block, right of=Iinput] (Idelta) {$\delta$};
  \node[block, right of=Idelta, node distance=1.3cm] (f) {$\inc{(\lift{R})}$};
  \node[block, right of=f, node distance=1.5cm] (S) {$\int$};
  \node[right of=S] (output)  {$O$};
  \node[block, below of=f, node distance=.9cm] (z) {$\zm$};
  \draw[->] (Iinput) -- (Idelta);
  \draw[->] (f) -- node (o) {} (S);
  \draw[->>] (S) -- (output);
  \draw[->>] (o) |- (z);
  \draw[->>] (z) -- (f);
  \draw[->>] (Idelta) -- (f);
\end{tikzpicture}
\end{center}
\vspace{-2ex}
This circuit implements \defined{semi-na\"{\i}ve evaluation}
(Algorithm~2 in~\cite{greco-sldm15}).  We have just proven the
correctness of semi-na\"{\i}ve evaluation as an immediate consequence
of the cycle rule!

%In \refsec{sec:recursive-example} we show a concrete example, applying Algorithm~\ref{algorithm-rec}
%to a recursive query computing the transitive closure of a graph.
