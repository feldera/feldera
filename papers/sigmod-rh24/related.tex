\section{Related work}\label{sec:related}

Incremental view maintenance~\cite{gupta-sigmod93, griffin-sigmod95,
  chaudhuri-icde95, gupta-idb95, chirkova-book12} is a much studied
problem in databases.  A survey of results for Datalog queries is
present in~\cite{motik-ai19}.  DBToaster introduced recursive
recursive IVM~\cite{ahmad-vldb09, koch-pods10}, where the
incrementalization process is repeated for the delta query.

Many custom algorithms were published for various classes of queries:
e.g.~\cite{koch-pods16} handles positive nested relational calculus.
DYN~\cite{idris-sigmod17} and IDYN~\cite{idris-vldb18, idris-sigmod19}
focus on acyclic conjunctive queries.  Instead of keeping the output
view materialized they build data structures that allow efficiently
querying the output views.  PAI maps~\cite{abeysinghe-sigmod22} are
specially designed for queries with correlated aggregations.
AJU~\cite{wang-sigmod20} focuses on foreign-key joins.  It is a matter
of future work to evaluate whether custom \dbsp operators can match
the efficiency of systems specialized for narrow classes of queries.

\dbsp is a bottom-up system, which always produces eagerly the
\emph{changes} to the output views.  Instead of maintaining the output
view entirely, \dbsp proposes generating deltas as the output of the
computation (similar to the kSQL~\cite{jafarpour-edbt19} \texttt{EMIT
  CHANGES} queries).  The idea that both inputs and outputs to an IVM
system are streams of changes seems trivial, but this is key to the
symmetry of our solution: both in our definition of IVM, and the
fundamental reason that the chain rule exists --- the chain rule is
the one that makes our structural induction IVM algorithm possible.

Several IVM algorithms for Datalog-like languages use counting based
approaches~\cite{Dewan-iis92,motik-aaai15} that maintain the number of derivations of each
output fact: DRed~\cite{gupta-sigmod93} and its variants~\cite{Ceri-VLDB91,Wolfson-sigmod91,
Staudt-vldb96,Kotowski-rr11,Lu-sigmod95,Apt-sigmod87}, the backward-forward algorithm
and variants~\cite{motik-aaai15,Harrison-wdd92,motik-ai19}.
\dbsp is more general,
and our incrementalization algorithm handles arbitrary recursive queries and
generates more efficient plans for recursive queries
in the presence of arbitrary updates (especially deletions, where competing approaches
may over-delete).  Interestingly, the \zrs weights in \dbsp are related
to the counting-number-of-derivations approaches, but our use of the $\distinct$
operator shows that precise counting is not necessary.

Picallo et al.~\cite{picallo-scop19} provide a general solution to IVM for
rich languages.  \dbsp requires a group structure on the values operated on;
this assumption has two major practical benefits: it simplifies the mathematics considerably
(e.g., Picallo uses monoid actions to model changes), and it provides a general, simple
algorithm (\ref{algorithm-inc}) for incrementalizing arbitrary programs.  The downside of
\dbsp is that one has to find a suitable group structure (e.g., \zrs for sets) to ``embed''
the computation.  Picallo's notion of ``derivative'' is not unique: they need creativity to choose
the right derivative definition, we need creativity to find the right group structure.

Finding a suitable group structure has proven easy for relations (both~\cite{koch-pods10}
and~\cite{green-tcs11} use \zrs to uniformly model data and insertions/deletions), but it is
not obvious how to do it for other data types, such as sorted collections, or tree-shaped
collections (e.g., XML or JSON documents)~\cite{foster-planx08}.  An intriguing question
is ``what other interesting group structures could this be applied to besides \zrs?''
Papers such as~\cite{nikolic-icmd18} explore other possibilities, such as matrix algebra,
linear ML models, or conjunctive queries.

%\dbsp does not do anything special for triangle queries~\cite{kara-tds20}.  Are there
%better algorithms for this case?

%In \secref{sec:extensions} we have briefly mentioned that \dbsp can easily
%model window and stream database queries~\cite{arasu-tr02,aurora}; it is an
%interesting question whether there are CQL queries that cannot be expressed in \dbsp
%(we conjecture that there aren't any).

\cite{bonifati-iclp2018} implemented a verified IVM algorithm for a particular
class of graph queries called Regular Datalog, with an implementation machine-checked in the
Coq proof assistant. Their focus is on a particular algorithm and the approach does not
consider other SQL operators, general recursion, or custom operators (although it is modular
in the sense that it works on any query by incrementalizing it recursively). Furthermore,
for all queries a deletion in the input change stream requires running the non-incremental
query to recover.  We formally verify the theorems in our paper, which
are much broader in scope, but not our implementations.

\dbsp is also related to Differential Dataflow
(DD)~\cite{mcsherry-cidr13, murray-sosp13} and its theoretical
foundations~\cite{abadi-fossacs15} (and
recently~\cite{mcsherry-vldb20,chothia-vldb16}).  DD's computational
model is more powerful than \dbsp, since it allows time values to be
part of an arbitrary lattice.  In fact, DD is the only other framework
which we are aware of that can incrementalize recursive queries as
efficiently as \dbsp does.  In contrast, our model uses either
``linear'' times, or nested time dimensions via the modular lifting
transformer ($\lift{}$).  \dbsp can express both incremental and
non-incremental computations.  Most importantly, \dbsp comes with
Algorithm~\ref{algorithm-inc}, a syntax-directed translation that can
convert any expressible query into an incremental version --- in DD
users have to assemble incremental queries manually using incremental
operators.  Materialize Inc. offers a product that automates
incrementalization for SQL queries based on DD.  Differential
Datalog~\cite{ryzhyk-datalog19} does it for a Datalog dialect.  Unlike
DD, \dbsp is a modular theory, which easily accommodates the addition
of new operators: as long as we can express a new operator as a \dbsp
circuit, we can (1) define its incremental version, (2) apply the
incrementalization algorithm to obtain an efficient incremental
implementation, and (3) be confident that it composes with any other
operators.  For example, \dbsp offers explicitly the delay operator
$\zm$, which can be very useful for writing many new classes of
programs.
